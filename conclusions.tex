\chapter{Discussion and Conclusions}

\emph{CRAXDroid} aims at exploiting vulnerabilities inside Android apps. To be
specific, programming logic that works fine under normal condition, but fails
when unexpected input are fed with. Android x86 is taken as the first Guinea
pig to show that \emph{CRAXDroid} does its works. However, most devices running
Android are built on ARM architecture, and exploiting software on ARM
architecture is quite different from exploiting software on x86 architecture,
since the two are designed differently. Raspberry Pi is taken to examine stack
overflow vulnerabilities and to study ARM calling convention. Taking advantage
from QEMU makes cross-platform testing easy. Two linux distributions that
support ARM are taken to improve \emph{CRAXDroid}. One is Debian ARM, which is
emulated by ARM926 (ARMv5) CPU, and the other one is Raspbian, which is
emulated by ARM1176 (ARMv6) CPU. Three programs are tested against the two
distributions, and two programs are successfully generated exploit on Raspbian,
while only one is generated on Debian ARM.

While \emph{CRAXDroid} development is still in its early stage, there are
several features to be implemented next

\begin{itemize}

\item{Emulating Android ARM with pure QEMU}

Although AOSP Android provides a emulator based on QEMU to emulate Android, the
emulator is highly customized, and might not fit for \emph{S$^{2}$E} to use. To
make Android ARM runs on pure QEMU, a customized kernel is needed, and the
installer from Android x86 could be brought in to help.

\item{Real Exploit For ARM}

Currently, we use the same exploit generate technique to generate exploit for
both x86 and ARM architecture. However, the generated exploit would not work on
ARM, since ARM is born with non-executable stack. More exploit techniques, such
as ROP (Return Oriented Programming), need to be brought in to generate
feasible exploit for both ARM and x86.

\item{Testing on Various ARM architecures}

There are many ARM architectures, from ARMv1 to ARMv8. It is uncleared if
\emph{CRAXDroid} works on all of them. Modern smart phones are equipped with
Cortex family CPUs, which are ARMv7 architectures. We should test ARMv7 as
well.

\end{itemize}
